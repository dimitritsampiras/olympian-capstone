\documentclass[12pt, titlepage]{article}

\usepackage{fullpage}
\usepackage[round]{natbib}
\usepackage{multirow}
\usepackage{booktabs}
\usepackage{tabularx}
\usepackage{graphicx}
\usepackage{float}
\usepackage{hyperref}
\hypersetup{
	colorlinks,
	citecolor=blue,
	filecolor=black,
	linkcolor=red,
	urlcolor=blue
}

\input{../../Comments}
\input{../../Common}

\newcounter{acnum}
\newcommand{\actheacnum}{AC\theacnum}
\newcommand{\acref}[1]{AC\ref{#1}}

\newcounter{ucnum}
\newcommand{\uctheucnum}{UC\theucnum}
\newcommand{\uref}[1]{UC\ref{#1}}

\newcounter{mnum}
\newcommand{\mthemnum}{M\themnum}
\newcommand{\mref}[1]{M\ref{#1}}

\begin{document}
	
	\title{System Design for \progname{}} 
	\author{\authname}
	\date{\today}
	
	\maketitle
	
	\pagenumbering{roman}
	
	\section{Revision History}
	
	\begin{tabularx}{\textwidth}{p{3cm}p{2cm}X}
		\toprule {\bf Date} & {\bf Version} & {\bf Notes}\\
		\midrule
		Date 1 & 1.0 & Notes\\
		Date 2 & 1.1 & Notes\\
		\bottomrule
	\end{tabularx}
	
	\newpage
	
	\section{Reference Material}
	
	This section records information for easy reference.
	
	\subsection{Abbreviations and Acronyms}
	
	\renewcommand{\arraystretch}{1.2}
	\begin{tabular}{l l} 
		\toprule		
		\textbf{symbol} & \textbf{description}\\
		\midrule 
		\progname & Explanation of program name\\
		\wss{...} & \wss{...}\\
		\bottomrule
	\end{tabular}\\
	
	\newpage
	
	\tableofcontents
	
	\newpage
	
	\listoftables
	
	\listoffigures
	
	\newpage
	
	\pagenumbering{arabic}
	
	\section{Introduction}
	
	\wss{Include references to your other documentation}
	
	\section{Purpose}
	
	\wss{Purpose of your design documentation}
	
	\wss{Point to your other design documents}
	
	\section{Scope}
	
	\wss{Include a figure that show the System Context (showing the boundary between
		your system and the environment around it.)}
	
	\section{Project Overview}
	
	\subsection{Normal Behaviour}
	
	\subsection{Undesired Event Handling}
	
	\wss{How you will approach undesired events}
	
	\subsection{Component Diagram}
	
	\subsection{Connection Between Requirements and Design} \label{SecConnection}
	
	\wss{The intention of this section is to document decisions that are made
		``between'' the requirements and the design.  To satisfy some requirements,
		design decisions need to be made.  Rather than make these decisions implicit,
		they are explicitly recorded here.  For instance, if a program has security
		requirements, a specific design decision may be made to satisfy those
		requirements with a password.}
	
	\section{System Variables}
	
	\wss{Include this section for Mechatronics projects}
	
	\subsection{Monitored Variables}
	
	\subsection{Controlled Variables}
	
	\subsection{Constants Variables}
	
	\section{User Interfaces}
	
	\wss{Design of user interface for software and hardware.  Attach an appendix if
		needed. Drawings, Sketches, Figma}
	
	\section{Design of Hardware}
	
	\wss{Most relevant for mechatronics projects}
	\wss{Show what will be acquired}
	\wss{Show what will be built, with detail on fabrication and materials}
	\wss{Include appendices as appropriate, possibly with sketches, drawings, CAD, etc}
	
	\section{Design of Electrical Components}
	
	\wss{Most relevant for mechatronics projects}
	\wss{Show what will be acquired}
	\wss{Show what will be built, with detail on fabrication and materials}
	\wss{Include appendices as appropriate, possibly with sketches, drawings,
		circuit diagrams, etc}
	
	\section{Design of Communication Protocols}
	
	\wss{If appropriate}
	
	\section{Timeline}
	
	\wss{Schedule of tasks and who is responsible}
	
	% \bibliographystyle {plainnat}
	% \bibliography{../../../refs/References}
	
	\newpage{}
	
	\appendix
	
	\section{Interface}
	
	\wss{Include additional information related to the appearance of, and
		interaction with, the user interface}
	
	\section{Mechanical Hardware}
	
	\section{Electrical Components}
	
	\section{Communication Protocols}
	
	\section{Reflection}
	
	The information in this section will be used to evaluate the team members on the
	graduate attribute of Problem Analysis and Design.  Please answer the following questions:
	
	\begin{enumerate}
		\item What are the limitations of your solution?  Put another way, given
		unlimited resources, what could you do to make the project better? (LO\_ProbSolutions)
		\item Give a brief overview of other design solutions you considered.  What
		are the benefits and tradeoffs of those other designs compared with the chosen
		design?  From all the potential options, why did you select documented design?
		(LO\_Explores)
	\end{enumerate}
	
\end{document}