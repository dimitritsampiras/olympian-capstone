\documentclass[12pt, titlepage]{article}

\usepackage{amsmath, mathtools}

\usepackage[round]{natbib}
\usepackage{amsfonts}
\usepackage{amssymb}
\usepackage{graphicx}
\usepackage{colortbl}
\usepackage{xr}
\usepackage{hyperref}
\usepackage{longtable}
\usepackage{xfrac}
\usepackage{tabularx}
\usepackage{float}
\usepackage{siunitx}
\usepackage{booktabs}
\usepackage{multirow}
\usepackage[section]{placeins}
\usepackage{caption}
\usepackage{fullpage}

\hypersetup{
bookmarks=true,     % show bookmarks bar?
colorlinks=true,       % false: boxed links; true: colored links
linkcolor=red,          % color of internal links (change box color with linkbordercolor)
citecolor=blue,      % color of links to bibliography
filecolor=magenta,  % color of file links
urlcolor=cyan          % color of external links
}

\usepackage{array}

\externaldocument{../../SRS/SRS}

\input{../../Comments}
\input{../../Common}

\begin{document}

\title{Module Interface Specification for \progname{}}

\author{\authname}

\date{\today}

\maketitle

\pagenumbering{roman}

\section{Revision History}

\begin{tabularx}{\textwidth}{p{3cm}p{2cm}X}
\toprule {\bf Date} & {\bf Version} & {\bf Notes}\\
\midrule
Date 1 & 1.0 & Notes\\
Date 2 & 1.1 & Notes\\
\bottomrule
\end{tabularx}

~\newpage

\section{Symbols, Abbreviations and Acronyms}

See SRS Documentation at \wss{give url}

\wss{Also add any additional symbols, abbreviations or acronyms}

\newpage

\tableofcontents

\newpage

\pagenumbering{arabic}

\section{Introduction}

The following document details the Module Interface Specifications for
\wss{Fill in your project name and description}

Complementary documents include the System Requirement Specifications
and Module Guide.  The full documentation and implementation can be
found at \url{...}.  \wss{provide the url for your repo}

\section{Notation}

\wss{You should describe your notation.  You can use what is below as
  a starting point.}

The structure of the MIS for modules comes from \citet{HoffmanAndStrooper1995},
with the addition that template modules have been adapted from
\cite{GhezziEtAl2003}.  The mathematical notation comes from Chapter 3 of
\citet{HoffmanAndStrooper1995}.  For instance, the symbol := is used for a
multiple assignment statement and conditional rules follow the form $(c_1
\Rightarrow r_1 | c_2 \Rightarrow r_2 | ... | c_n \Rightarrow r_n )$.

The following table summarizes the primitive data types used by \progname. 

\begin{center}
\renewcommand{\arraystretch}{1.2}
\noindent 
\begin{tabular}{l l p{7.5cm}} 
\toprule 
\textbf{Data Type} & \textbf{Notation} & \textbf{Description}\\ 
\midrule
character & char & a single symbol or digit\\
integer & $\mathbb{Z}$ & a number without a fractional component in (-$\infty$, $\infty$) \\
natural number & $\mathbb{N}$ & a number without a fractional component in [1, $\infty$) \\
real & $\mathbb{R}$ & any number in (-$\infty$, $\infty$)\\
\bottomrule
\end{tabular} 
\end{center}

\noindent
The specification of \progname \ uses some derived data types: sequences, strings, and
tuples. Sequences are lists filled with elements of the same data type. Strings
are sequences of characters. Tuples contain a list of values, potentially of
different types. In addition, \progname \ uses functions, which
are defined by the data types of their inputs and outputs. Local functions are
described by giving their type signature followed by their specification.

\section{Module Decomposition}

The following table is taken directly from the Module Guide document for this project.

\begin{table}[h!]
\centering
\begin{tabular}{p{0.3\textwidth} p{0.6\textwidth}}
\toprule
\textbf{Level 1} & \textbf{Level 2}\\
\midrule

{Hardware-Hiding} & ~ \\
\midrule

\multirow{7}{0.3\textwidth}{Behaviour-Hiding} & Input Parameters\\
& Output Format\\
& Output Verification\\
& Temperature ODEs\\
& Energy Equations\\ 
& Control Module\\
& Specification Parameters Module\\
\midrule

\multirow{3}{0.3\textwidth}{Software Decision} & {Sequence Data Structure}\\
& ODE Solver\\
& Plotting\\
\bottomrule

\end{tabular}
\caption{Module Hierarchy}
\label{TblMH}
\end{table}

\newpage
~\newpage

\section{MIS of Project Olympian} \label{Template} 

\section{Template Module}

Template

\subsection{Uses}

\subsection{Syntax}

\subsubsection{Exported Constants}

\subsubsection{Exported Access Programs}

\begin{center}
\begin{tabular}{p{2cm} p{4cm} p{4cm} p{2cm}}
\hline
\textbf{Name} & \textbf{In} & \textbf{Out} & \textbf{Exceptions} \\
\hline
\wss{accessProg} & - & - & - \\
\hline
\end{tabular}
\end{center}

\subsection{Semantics}

\subsubsection{State Variables}

\wss{Not all modules will have state variables.  State variables give the module
  a memory.}

\subsubsection{Environment Variables}

\wss{This section is not necessary for all modules.  Its purpose is to capture
  when the module has external interaction with the environment, such as for a
  device driver, screen interface, keyboard, file, etc.}

\subsubsection{Assumptions}

\wss{Try to minimize assumptions and anticipate programmer errors via
  exceptions, but for practical purposes assumptions are sometimes appropriate.}

\subsubsection{Access Routine Semantics}

\noindent \wss{accessProg}():
\begin{itemize}
\item transition: \wss{if appropriate} 
\item output: \wss{if appropriate} 
\item exception: \wss{if appropriate} 
\end{itemize}

\wss{A module without environment variables or state variables is unlikely to
  have a state transition.  In this case a state transition can only occur if
  the module is changing the state of another module.}

\wss{Modules rarely have both a transition and an output.  In most cases you
  will have one or the other.}

\subsubsection{Local Functions}

\wss{As appropriate} \wss{These functions are for the purpose of specification.
  They are not necessarily something that is going to be implemented
  explicitly.  Even if they are implemented, they are not exported; they only
  have local scope.}

\newpage

\section{Fitess Unit Of Measurement Type Module}

\subsection{Uses}

\subsection{Syntax}

\subsubsection{Exported Types}
FitessUnit = \{Set, Rep, Curl, Press, Jump, Step, Stretch, Push, Pull, Second, Minute, Hour, Day...\} 
\\
\textit{//Many more, just stating a few}
\subsubsection{Exported Access Programs}

None

\subsection{Semantics}
\subsubsection{State Variables}
None

\subsubsection{Environment Variables}
None

\subsubsection{Assumptions}
None

\subsubsection{Access Routine Semantics}

None

\subsubsection{Local Functions}


\newpage

\section{Quantifier Module}

\subsection{Uses}

\subsection{Syntax}

\subsubsection{Exported Constants}
Quantifier = ?
\subsubsection{Exported Access Programs}

\begin{center}
	\begin{tabular}{p{2cm} p{4cm} p{4cm} p{2cm}}
		\hline
		\textbf{Name} & \textbf{In} & \textbf{Out} & \textbf{Exceptions} \\
		\hline
		Quantifier & $FitnessUnit, \mathbb{Q}$ & Quantifier &  \\
		getUnit &  & $FitnessUnit$ &  \\
		getValue &  & $\mathbb{Q}$ &  \\
		\hline
	\end{tabular}
\end{center}

\subsection{Semantics}

\subsubsection{State Variables}
unit: $FitnessUnit$ \\
value: $\mathbb{Q}$

\subsubsection{Environment Variables}

\subsubsection{Assumptions}
None

\subsubsection{Access Routine Semantics}

Quantifier(u, v):
\begin{itemize}
	\item transition: $unit, value := u, v$
	\item output: $out := self$
	\item exception: None
\end{itemize}

unit():
\begin{itemize}
	\item output: $out := unit$
	\item exception: None
\end{itemize}

value():
\begin{itemize}
	\item output: $out := value$
	\item exception: None
\end{itemize}

\subsubsection{Local Functions}


\newpage
% Exercise 

\section{Exercise Module}

\subsection{Uses}
N/A

\subsection{Syntax}

\subsubsection{Exported Constants}
Exercise = ?
\subsubsection{Exported Access Programs}

\begin{center}
	\begin{tabular}{p{2cm} p{4cm} p{4cm} p{2cm}}
		\hline
		\textbf{Name} & \textbf{In} & \textbf{Out} & \textbf{Exceptions} \\
		\hline
		Exercise & Quantifier, String & Exercise &  \\
		getQuantifier &  & Quantifier &  \\
		getDescription & & String & \\
		\hline
	\end{tabular}
\end{center}

\subsection{Semantics}

\subsubsection{State Variables}

quantifier: $Quantifier$ \\
description: $String$

\subsubsection{Environment Variables}

\subsubsection{Assumptions}

None

\subsubsection{Access Routine Semantics}

\noindent Exercise(q, d):
\begin{itemize}
	\item transition: $quantifier, description := q, d$
	\item output: $out := self$
	\item exception: None
\end{itemize}

\noindent quantifier():
\begin{itemize}
	\item output: $out := quantifier$
	\item exception: None
\end{itemize}

\noindent description():
\begin{itemize}
	\item output: $out := description$
	\item exception: None
\end{itemize}

\subsubsection{Local Functions}

\newpage

\section{Timed Sequence Module}

\subsection{Uses}

\subsection{Syntax}

\subsubsection{Exported Constants}
TimedSequence(T) = ?
\subsubsection{Exported Access Programs}

\begin{center}
	\begin{tabular}{p{4cm} p{4cm} p{4cm} p{2cm}}
		\hline
		\textbf{Name} & \textbf{In} & \textbf{Out} & \textbf{Exceptions} \\
		\hline
		TimedSequence & seq of T, String, Time & TimedSequence &  \\
		getSeq &  & seq of Exercise &  \\
		getDescription &  & String &  \\
		getDuration &  & Time &  \\
		add & T & &  \\
		\hline
	\end{tabular}
\end{center}

\subsection{Semantics}

\subsubsection{State Variables}
sequence: $T[]$ \\
description: $String$ \\
duration: $Time$ \\

\subsubsection{Environment Variables}

\subsubsection{Assumptions}

\subsubsection{Access Routine Semantics}

TimedSequence(s, desc, dur):
\begin{itemize}
	\item transition: $sequence, description, duration := s, desc, dur$
	\item output: $out := self$
	\item exception: None
\end{itemize}

sequence():
\begin{itemize}
	\item output: $out := sequence$
	\item exception: None
\end{itemize}

description():
\begin{itemize}
	\item output: $out := description$
	\item exception: None
\end{itemize}

duration():
\begin{itemize}
	\item output: $out := duration$
	\item exception: None
\end{itemize}

add(t):
\begin{itemize}
	\item transition: $sequence.put(t)$
	\item exception: None
\end{itemize}

\subsubsection{Local Functions}

\newpage

\section{Workout Module}

\subsection{Uses}
TimedSequence(Exercise)
\subsection{Syntax}

\subsubsection{Exported Constants}
Workout = ?
\subsubsection{Exported Access Programs}

\begin{center}
	\begin{tabular}{p{2cm} p{4cm} p{4cm} p{2cm}}
		\hline
		\textbf{Name} & \textbf{In} & \textbf{Out} & \textbf{Exceptions} \\
		\hline
		Workout & seq of Exercise, String, Time & Workout &  \\
		getExercises &  & seq of Exercise &  \\
		getDescription &  & String &  \\
		getDuration &  & Time &  \\
		addExercise & Exercise & &  \\
		\hline
	\end{tabular}
\end{center}

\subsection{Semantics}

\subsubsection{State Variables}
exercises: $Exercise[]$ \\
description: $String$ \\
duration: $Time$ \\

\subsubsection{Environment Variables}

\subsubsection{Assumptions}

\subsubsection{Access Routine Semantics}

Workout(e, desc, dur):
\begin{itemize}
	\item transition: $exercises, description, duration := e, desc, dur$
	\item output: $out := self$
	\item exception: None
\end{itemize}

exercise():
\begin{itemize}
	\item output: $out := exercise$
	\item exception: None
\end{itemize}

description():
\begin{itemize}
	\item output: $out := description$
	\item exception: None
\end{itemize}

duration():
\begin{itemize}
	\item output: $out := duration$
	\item exception: None
\end{itemize}

AddExercise(e):
\begin{itemize}
	\item transition: $exercises.put(e)$
	\item exception: None
\end{itemize}

\subsubsection{Local Functions}

\newpage

\section{Workout Routine Module}

\subsection{Uses}
TimedSequence(Workout)
\subsection{Syntax}

\subsubsection{Exported Constants}
Routine = ? \\
\textit{//Note: Same as a Workout Routine, just short version}
\subsubsection{Exported Access Programs}

\begin{center}
	\begin{tabular}{p{2cm} p{4cm} p{4cm} p{2cm}}
		\hline
		\textbf{Name} & \textbf{In} & \textbf{Out} & \textbf{Exceptions} \\
		\hline
		Routine & seq of Workout, Duration, Description & Routine &  \\
		getWorkouts &  & seq of Workout &  \\
		getDescription &  & String &  \\
		getDuration &  & Time &  \\
		\hline
	\end{tabular}
\end{center}

\subsection{Semantics}

\subsubsection{State Variables}
workouts: $Exercise[]$ \\
description: $String$ \\
duration: $Time$ \\

\subsubsection{Environment Variables}

\subsubsection{Assumptions}

\subsubsection{Access Routine Semantics}

Routine(w, desc, dur):
\begin{itemize}
	\item transition: $workouts, description, duration := w, desc, dur$
	\item output: $out := self$
	\item exception: None
\end{itemize}

exercise():
\begin{itemize}
	\item output: $out := exercise$
	\item exception: None
\end{itemize}

description():
\begin{itemize}
	\item output: $out := description$
	\item exception: None
\end{itemize}

duration():
\begin{itemize}
	\item output: $out := duration$
	\item exception: None
\end{itemize}

AddWorkout(w):
\begin{itemize}
	\item transition: $workouts.put(w)$
	\item exception: None
\end{itemize}

\subsubsection{Local Functions}

\newpage

\section{User Profile Module}

\subsection{Uses}

\subsection{Syntax}

\subsubsection{Exported Constants}
User = ?
\subsubsection{Exported Access Programs}

\begin{center}
	\begin{tabular}{p{4cm} p{4cm} p{3cm} p{2cm}}
		\hline
		\textbf{Name} & \textbf{In} & \textbf{Out} & \textbf{Exceptions} \\
		\hline
		User & String, String, String, String & User &  \\
		username &  & String &  \\
 		password &  & String &  \\
 		email &  & String &  \\
 		nickname &  & String &  \\
 		fitness Goals &  & seq of FitnessGoal &  \\
 		created Workouts &  & seq of Workout &  \\
 		created Routines &  & seq of Routine &  \\
 		saved Workouts &  & seq of Workout &  \\
 		saved Routines &  & seq of Routine &  \\
		\hline
	\end{tabular}
\end{center}

\subsection{Semantics}

\subsubsection{State Variables}
username: $String$ \\
password: $String$ \\
email: $String$ \\
nickname: $String$ \\
fitnessGoals: $seq of FitnessGoal$ \\
created Workouts: $Workout[]$ \\
created Routines: $Routine[]$ \\
saved Workouts: $Workout[]$ \\
saved Routines: $Routine[]$ \\

\subsubsection{Environment Variables}

\subsubsection{Assumptions}

\subsubsection{Access Routine Semantics}

User(username, pass, email, nickname):
\begin{itemize}
	\item transition: $username, password, email, nickname := username, pass, email, nickname$
	\item output: $out := self$
	\item exception: None
\end{itemize}

username():
\begin{itemize}
	\item output: $out := username$
	\item exception: None
\end{itemize}

email():
\begin{itemize}
	\item output: $out := email$
	\item exception: PremissionException
\end{itemize}

password():
\begin{itemize}
	\item output: $out := password$
	\item exception: PremissionException
\end{itemize}

nickname():
\begin{itemize}
	\item output: $out := nickname$
	\item exception: None
\end{itemize}

fitnessGoals():
\begin{itemize}
	\item output: $out := fitnessGoals$
	\item exception: None
\end{itemize}

createdWorkouts():
\begin{itemize}
	\item output: $out := createdWorkouts$
	\item exception: None
\end{itemize}

createdRoutines():
\begin{itemize}
	\item output: $out := createdRoutines$
	\item exception: None
\end{itemize}

savedWorkouts():
\begin{itemize}
	\item output: $out := savedWorkouts$
	\item exception: None
\end{itemize}

savedRoutines():
\begin{itemize}
	\item output: $out := savedRoutines$
	\item exception: None
\end{itemize}

createGoal(g):
\begin{itemize}
	\item transition: $ fitnessGoals.put(g) $
	\item exception: None
\end{itemize}

createWorkout(workout):
\begin{itemize}
	\item transition: $ createdWorkouts.put(w), \ savedWorkouts.put(w) $
	\item exception: None
\end{itemize}

createRoutine(routine):
\begin{itemize}
	\item transition: $ createdRoutine.put(r), \ savedRoutines.put(r) $
	\item exception: None
\end{itemize}

saveWorkout(w):
\begin{itemize}
	\item transition: $ savedWorkouts.put(w) $
	\item exception: None
\end{itemize}

saveRoutine(r):
\begin{itemize}
	\item transition: $ savedRoutines.put(r) $
	\item exception: None
\end{itemize}

\subsubsection{Local Functions}

\newpage

\section{User Login Module}

\subsection{Uses}

\subsection{Syntax}

\subsubsection{Exported Constants}
UserLogin = ?
\subsubsection{Exported Access Programs}

\begin{center}
	\begin{tabular}{p{2cm} p{4cm} p{4cm} p{2cm}}
		\hline
		\textbf{Name} & \textbf{In} & \textbf{Out} & \textbf{Exceptions} \\
		\hline
		UserLogin & Database & UserLogin &  \\
		attemptLogin & String, String & $\mathbb{B}$ & tooManyAttempts \\
		validateUser & String, String & $\mathbb{B}$ &  \\
		\hline
	\end{tabular}
\end{center}

\subsection{Semantics}

\subsubsection{State Variables}
attempts: $\mathbb{Z}$\\
maxAttempts: $\mathbb{Z}$\\
userDatabase: $Database$ \\
\subsubsection{Environment Variables}

\subsubsection{Assumptions}
There exists a stored number for maximum attempts.
\subsubsection{Access Routine Semantics}

\noindent UserLogin(db):
\begin{itemize}
	\item transition: $userDatabase, attempts := db, 0$
	\item output: $out := self$
	\item exception: None
\end{itemize}

\noindent attemptLogin(user, pass):
\begin{itemize}
	\item transition: $attempts := attempts + 1$
	\item output: $out := validateUser(user, pass) \land userDatabase.getUser(user) = (pass)$
	\item exception: $exc := attempts \ge maxAttempts \Rightarrow tooManyAttempts$
\end{itemize}

\subsubsection{Local Functions}

\noindent validateUser(user, pass) $\rightarrow \mathbb{B}$
\begin{itemize}
	\item output: $out := userDatabase.validate(user, pass)$
	\item exception: None
\end{itemize}
\newpage

\section{User Registration Module}

\subsection{Uses}

\subsection{Syntax}

\subsubsection{Exported Constants}
userRegistration = ?
\subsubsection{Exported Access Programs}

\begin{center}
	\begin{tabular}{p{4cm} p{4cm} p{4cm} p{4cm}}
		\hline
		\textbf{Name} & \textbf{In} & \textbf{Out} & \textbf{Exceptions} \\
		\hline
		userRegistration & Database & userRegistration &  \\
		register & String, String, String, String & String &  \\
		exists & String, String & $\mathbb{B}$ & \\
		validate & String, String, String, String & $\mathbb{B}$ & invalidPassword, invalidUsername, invalidEmail\\
		\hline
	\end{tabular}
\end{center}

\subsection{Semantics}

\subsubsection{State Variables}
userDatabase: $Database$

\subsubsection{Environment Variables}

\subsubsection{Assumptions}
Regex operations are allowed on strings
\subsubsection{Access Routine Semantics}

\noindent userRegistration(db):
\begin{itemize}
	\item transition: $userDatabase := db$
	\item output: $out := self$
	\item exception: None
\end{itemize}

\noindent register(username, password, email, nickname):
\begin{itemize}
	\item output: $out:= \lnot exists(username, email) \land validate(username, password, email, nickname)$
	\item exception: None
\end{itemize}

\subsubsection{Local Functions}

\noindent exists(username, email)$\rightarrow \mathbb{B}$:
\begin{itemize}
	\item output: $out := userDatabase.existsUser(username) \lor userDatabase.existsEmail(email)$
	\item exception: None
\end{itemize}

\noindent validate(username, email, password, nickname)$\rightarrow \mathbb{B}$:
\begin{itemize}
	\item output: $out := username.matches(\{6,20\}) \\ \land password.matches( \land (?=.*\backslash d)(?=.*[a-z])(?=.*[A-Z]).\{6,20\}\$ ) \\ \land email.matches(/ \land \backslash w+([ \backslash .-]? \backslash w+)*@ \backslash w+([\backslash.-]?\backslash w+)*(\backslash.\backslash w{2,3})+\$/) \\ \land nickname.len() \ge 1$
	\item exception: $exc := \lnot username.matches(\{6,20\}) \Rightarrow$ invalidUsername
	\item exception: $exc := \lnot email.matches(/ \land \backslash w+([ \backslash .-]? \backslash w+)*@ \backslash w+([\backslash.-]?\backslash w+)*(\backslash.\backslash w{2,3})+\$/) \Rightarrow$ invalidEmail
	\item exception: $exc := \lnot password.matches( \land (?=.*\backslash d)(?=.*[a-z])(?=.*[A-Z]).\{6,20\}\$ ) \Rightarrow$ invalidPassword 
	\item exception: $exc := nickname.len() < 1 \Rightarrow$ invalidNickname
\end{itemize}

\section{User Fitness Goal Module}

\subsection{Uses}

\subsection{Syntax}

\subsubsection{Exported Constants}
FitnessGoal = ?
\subsubsection{Exported Access Programs}

\begin{center}
	\begin{tabular}{p{2cm} p{4cm} p{4cm} p{2cm}}
		\hline
		\textbf{Name} & \textbf{In} & \textbf{Out} & \textbf{Exceptions} \\
		\hline
		FitnessGoal & String, Quantifier & FitnessGoal &  \\
		quantifier &  & Quantifier &  \\
		description &  & String &  \\
		currentProgress &  & Quantifier &  \\
		status &  & $\mathbb{B}$ &  \\
		progressGoal & Quantifier &  &  \\
		&  &  &  \\
		\hline
	\end{tabular}
\end{center}

\subsection{Semantics}

\subsubsection{State Variables}
quantifier: $Quantifier$ \\
description: $String$ \\
progress: $Quantifier$ \\

\subsubsection{Environment Variables}

\subsubsection{Assumptions}

\subsubsection{Access Routine Semantics}

\newpage

\noindent FitnessGoal(quant, desc):
\begin{itemize}
	\item transition: $quantifier, description, progression = quant, desc, Quantifier(quant.unit, 0)$
	\item output: $out := self$
	\item exception: None
\end{itemize}

\noindent quantifier():
\begin{itemize}
	\item output: $out := quantifier$
	\item exception: None
\end{itemize}

\noindent description():
\begin{itemize}
	\item output: $out := description$
	\item exception: None
\end{itemize}

\noindent currentProgress():
\begin{itemize}
	\item output: $out := progress$
	\item exception: None
\end{itemize}

\noindent status():
\begin{itemize}
	\item output: $out := progress.value < quantifier.value $
	\item exception: None
\end{itemize}

\noindent progressGoal(val):
\begin{itemize}
	\item transition: $progress.value := val$
	\item exception: None
\end{itemize}

\subsubsection{Local Functions}


\newpage

\section{Workout Browsing Module}

\subsection{Uses}

\subsection{Syntax}

\subsubsection{Exported Constants}
Browser = ?
\subsubsection{Exported Access Programs}

\begin{center}
	\begin{tabular}{p{2cm} p{4cm} p{4cm} p{2cm}}
		\hline
		\textbf{Name} & \textbf{In} & \textbf{Out} & \textbf{Exceptions} \\
		\hline
		&  &  &  \\
		\hline
	\end{tabular}
\end{center}

\subsection{Semantics}

\subsubsection{State Variables}

\subsubsection{Environment Variables}

\subsubsection{Assumptions}

\subsubsection{Access Routine Semantics}

\noindent \wss{accessProg}():
\begin{itemize}
	\item transition:
	\item output:
	\item exception:
\end{itemize}

\subsubsection{Local Functions}

\newpage

\section{Creation Module}

\subsection{Uses}

\subsection{Syntax}

\subsubsection{Exported Constants}

\subsubsection{Exported Access Programs}

\begin{center}
	\begin{tabular}{p{2cm} p{4cm} p{4cm} p{2cm}}
		\hline
		\textbf{Name} & \textbf{In} & \textbf{Out} & \textbf{Exceptions} \\
		\hline
		&  &  &  \\
		\hline
	\end{tabular}
\end{center}

\subsection{Semantics}

\subsubsection{State Variables}

\subsubsection{Environment Variables}

\subsubsection{Assumptions}

\subsubsection{Access Routine Semantics}

\noindent \wss{accessProg}():
\begin{itemize}
	\item transition:
	\item output:
	\item exception:
\end{itemize}

\subsubsection{Local Functions}

\newpage

\bibliographystyle {plainnat}
\bibliography {../../../refs/References}

\newpage
\section{Exercise Creation Module}

\subsection{Uses}

\subsection{Syntax}

\subsubsection{Exported Constants}

\subsubsection{Exported Access Programs}

\begin{center}
	\begin{tabular}{p{2cm} p{4cm} p{4cm} p{2cm}}
		\hline
		\textbf{Name} & \textbf{In} & \textbf{Out} & \textbf{Exceptions} \\
		\hline
		&  &  &  \\
		\hline
	\end{tabular}
\end{center}

\subsection{Semantics}

\subsubsection{State Variables}

\subsubsection{Environment Variables}

\subsubsection{Assumptions}

\subsubsection{Access Routine Semantics}

\noindent \wss{accessProg}():
\begin{itemize}
	\item transition:
	\item output:
	\item exception:
\end{itemize}

\subsubsection{Local Functions}

\newpage

\bibliographystyle {plainnat}
\bibliography {../../../refs/References}

\newpage

\section{Workout Creation Module}

\subsection{Uses}

\subsection{Syntax}

\subsubsection{Exported Constants}

\subsubsection{Exported Access Programs}

\begin{center}
	\begin{tabular}{p{2cm} p{4cm} p{4cm} p{2cm}}
		\hline
		\textbf{Name} & \textbf{In} & \textbf{Out} & \textbf{Exceptions} \\
		\hline
		&  &  &  \\
		\hline
	\end{tabular}
\end{center}

\subsection{Semantics}

\subsubsection{State Variables}

\subsubsection{Environment Variables}

\subsubsection{Assumptions}

\subsubsection{Access Routine Semantics}

\noindent \wss{accessProg}():
\begin{itemize}
	\item transition:
	\item output:
	\item exception:
\end{itemize}

\subsubsection{Local Functions}

\newpage

\bibliographystyle {plainnat}
\bibliography {../../../refs/References}

\newpage

\section{Workout Routine Creation Module}

\subsection{Uses}

\subsection{Syntax}

\subsubsection{Exported Constants}

\subsubsection{Exported Access Programs}

\begin{center}
	\begin{tabular}{p{2cm} p{4cm} p{4cm} p{2cm}}
		\hline
		\textbf{Name} & \textbf{In} & \textbf{Out} & \textbf{Exceptions} \\
		\hline
		&  &  &  \\
		\hline
	\end{tabular}
\end{center}

\subsection{Semantics}

\subsubsection{State Variables}

\subsubsection{Environment Variables}

\subsubsection{Assumptions}

\subsubsection{Access Routine Semantics}

\noindent \wss{accessProg}():
\begin{itemize}
	\item transition:
	\item output:
	\item exception:
\end{itemize}

\subsubsection{Local Functions}

\newpage

\bibliographystyle {plainnat}
\bibliography {../../../refs/References}

\newpage

\section{Appendix} \label{Appendix}

\wss{Extra information if required}

\end{document}