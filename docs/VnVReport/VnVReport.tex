\documentclass[12pt, titlepage]{article}

\usepackage{booktabs}
\usepackage{tabularx}
\usepackage{hyperref}
\usepackage{changepage}
\usepackage{float}
\hypersetup{
    colorlinks,
    citecolor=black,
    filecolor=black,
    linkcolor=red,
    urlcolor=blue
}
\usepackage[round]{natbib}

\input{../Comments}
\input{../Common}

\begin{document}

\title{Verification and Validation Report: \progname} 
\author{\authname}
\date{\today}
	
\maketitle

\pagenumbering{roman}

\section{Revision History}

\begin{tabularx}{\textwidth}{p{3cm}p{2cm}X}
\toprule {\bf Date} & {\bf Version} & {\bf Notes}\\
\midrule
March 8 2023 & 1.0 & Initial draft\\
April 5 2023 & 2.0 & Final draft\\
\bottomrule
\end{tabularx}

~\newpage

\section{Reference Material}

This section records information for easy reference.

\subsection{Abbreviations and Acronyms}

\renewcommand{\arraystretch}{1.2}
\begin{tabular}{l l} 
	\toprule		
	\textbf{symbol} & \textbf{description}\\
	\midrule 
	AC & Anticipated Change\\
	iOS & \href{https://en.wikipedia.org/wiki/IOS}{Apple's proprietary operating system} \\
	M & Module \\
	MG & Module Guide \\
	MIS & Module Interface Specification \\
	OS & Operating System \\
	R & Requirement\\
	RPE & \href{https://en.wikipedia.org/wiki/Rating_of_perceived_exertion}{Rating of Perceived Exertion} \\
	SRS & Software Requirements Specification\\
	\progname & The Capstone course this project belongs to \\
	UC & Unlikely Change \\
	\bottomrule
\end{tabular}\\

\newpage

\tableofcontents

\listoftables %if appropriate

\newpage

\pagenumbering{arabic}

\subsection{Relevant Documentation}

\begin{enumerate}
	\item \href{https://github.com/dimitritsampiras/olympian/blob/main/docs/VnVPlan/VnVPlan.pdf}{VnV Plan}
	\item \href{https://github.com/dimitritsampiras/olympian/blob/main/docs/SRS/SRS.pdf}{SRS} 
	\item \href{https://github.com/dimitritsampiras/olympian/blob/main/docs/Design/MG/MG.pdf}{MG} 
	\item \href{https://github.com/dimitritsampiras/olympian/blob/main/docs/Design/MIS/MIS.pdf}{MIS} 	
\end{enumerate}

This document serves to record the results of testing and validation performed on the Olympian system, as well as any design decisions or changes made due to the results of said testing.

\section{User Testing Summary}
Unless Specified these tests were performed with a semi-structured interview where users were brought scenarios then asked questions relevant to that scenario. Most Scenarios followed the key use cases of the application.\\
Examples include:
\begin{enumerate}
	\item Creating an Account.
	\item Logging into an Account.
	\item Creating a Program.
	\item Creating a Workout.
	\item Creating an Exercise.
	\item Searching for an Exercise.
	\item Using the Discovery page to browse Workouts and Programs.
	\item Starting a Workout (as if the user were to use the application during their Workout).
\end{enumerate}

See the raw user test results in the appendix.

\section{Functional Requirements Evaluation}


\noindent During the user tests, users were asked to preform extra actions based on the FR test cases displayed in the VnV Plan.
Some example actions included: \\
\begin{enumerate}
    \item During workout creation, could you make your workout private?
    \item Could you try failing your password?
    \item Asking the user to attempt all actions when starting a workout.
\end{enumerate}


\noindent As for the non-user tests, manual unit tests were preformed in order to measure metrics.\\


\noindent The summary of the test cases are displayed below:


\begin{enumerate}
   
\subsubsection{Workout Routine Tests}
    \item{\textbf{test-WR-1}}: A Workout Routine can be created.\\
    Input: User inputs required data for a workout routine.
	
    Expected Output: A workout routine is stored in the database and is accessible to the user that created it. If the user determined that the workout to be public then it should be publicly visible.
    
    \textbf{Test result}: Workout routines can be created and are added to an existing Program. [Passed]
   
    \item{\textbf{test-WR-2}}: Editing a Workout Routine.\\
    Input: User edits a workout routine with new values.
	
    Expected Output: A workout routine is updated with new values in the database and new values are visible to accessible users.
    
    \textbf{Test result}: Workout routines can be edited, including changing the names, tags, and amending and removing exercises. [Passed]
   
\subsubsection{Exercise Tests}
    \item{\textbf{test-EX-1}}: Adding an Exercise to a Workout Routine.\\
    Input: A user created exercise with the parameters required for the exercise.

    Expected Output: The workout routine should include the added exercise.
    
    \textbf{Test result}: Users can add Exercises to Workout Routines and customize reps and sets. [Passed]
   
    \item{\textbf{test-EX-2}}: Removing an Exercise from a Workout Routine.\\
    Input: The user chooses to remove an exercise.
	
    Expected Output: Some feedback indicating that an exercise has been removed. The workout routine no longer contains the removed exercise.
    
    \textbf{Test result}: Users can remove exercises from Workout Routines. [Passed]
   
    \item{\textbf{test-EX-3}}: Limiting Exercises to a Workout Routine.\\
    Input: The number of exercises required to reach the limit of exercises per workout routine. 
	
    Expected Output: A notification to the user, before the limiting exercise notifying them of the limit, and preventing them from adding another exercise.
    
    \textbf{Test result}: Users can limit exercises to workout routines. [Passed]
   
\subsubsection{Quantifier Tests}
    \item{\textbf{test-QT-1}}: Adding Quantifiers to an Exercise.\\
    Input: A number and unit of measurement describing an exercise.
	
    Expected Output: The exercise now holds the given quantifier and is displayed to the user.
    
    \textbf{Test result}: Users can add quantifiers to their exercises. [Passed]
   
    \item{\textbf{test-QT-2}}: Removing Quantifiers from an Exercise.\\
    Input: Removal of a quantifier.
	
    Expected Output: The exercise no longer holds a quantifier and is not displayed to the user.
    
    \textbf{Test result}: Users can remove quantifiers from their exercises. [Passed]
   
    \item{\textbf{test-QT-3}}: Editing Quantifiers of an Exercise.\\
    Input: A number and unit of measurement describing an exercise.

    Expected Output: The exercise now holds the updated quantifier.
    
    \textbf{Test result}: Users can edit their quantifiers. [Passed]
   
\subsubsection{Publicity Tests}
    \item{\textbf{test-PB-1}}: Publicizing a Workout Routine.\\
    Input: An edit or addition to a workout routine to make it public.
	
    Expected Output: The workout routine is declared public in the database and is now visible to all users.
    
    \textbf{Test result}: Users can publicize a workout routine. [Passed]
   
    \item{\textbf{test-PB-2}}: Privatizing a Workout Routine.\\
    Input: An edit or addition to a workout routine to make is private.
	
    Expected Output: The workout routine is declared private in the database and is no longer visible to any user except the creator.
    
    \textbf{Test result}: Users can privatize a workout routine. [Passed]
   
\subsubsection{Workout Routine Saving Tests}
    \item{\textbf{test-WS-1}}: Saving a Public Workout Routine.\\
    Input: The other user workout routine is saved.
	
    Expected Output: The workout routine is now visible and accessible under the saved workout routines.
    
    \textbf{Test result}: Users can save a public workout routine. [Passed]
   
\subsubsection{Browsing Workout Routine Tests}
    \item{\textbf{test-BS-1}}: Browsing Workout Routines.\\
    Input: navigation movements to view the public workout routines.
	
    Expected Output: Multiple public workout routines should be displayed.
     
    \textbf{Test result}: Users can browse public programs according to specific categories. [Passed]
   
    \item{\textbf{test-BS-2}}: Search Workout Routines.\\
    Input: Search string inputs to view the public workout routines.
	
    Expected Output: Public workout routines that match the search criteria should be displayed. 
    
    \textbf{Test result}: Users can search publicly published routines using custom search text. [Passed]
   
\subsubsection{User Profile Tests}
    \item{\textbf{test-UP-1}}: Creating a User Profile.\\
    Input: The required parameters for creating a profile.
	
    Expected Output: The created profile is stored in a user database and the user should be logged into their profile.
    
    \textbf{Test result}: Users can create their profiles when signing up. [Passed]
   
    \item{\textbf{test-UP-2}}: Viewing Other User Profile.\\
    Input: Search criteria for the searched user profile.
	
    Expected Output: A user profile is displayed with their public routines, fitness goals, and other public profile data.
    
    \textbf{Test result}: Users can view other user profiles. [Passed]
   
\subsubsection{Fitness Goal Tests}
    \item{\textbf{test-FG-1}}: Creating a Fitness Goal.\\
    Input: Progress points towards a given fitness goal at a given date.
	
    Expected Output: The progress displayed towards a fitness goal should be visually updated and numerically updated in the database with a date
    
    \textbf{Test result}: Users can create a fitness goal. [Passed]
   
    \item{\textbf{test-FG-2}}: Progressing a Fitness Goal.\\
    Input: Progress points towards a given fitness goal at a given date.
	
    Expected Output: The progress displayed towards a fitness goal should be visually updated and numerically updated in the database with a date
    
    \textbf{Test result}: Users can progress fitness goals. [Passed]
 
\end{enumerate}

\subsection{Functional Requirements Results}

Tests Passed: 17\\
Tests Failed: 0\\
Total Tests: 17\\

\section{Nonfunctional Requirements Evaluation}

\noindent After users were brought through the test case scenarios they were asked questions based on the NFR test cases displayed in the VnV Plan.
Some example questions included: \\
\begin{enumerate}
	\item How did you find the speed of the application? Did you notice any latency or lagging?
	\item Any time the application prompted you to do an action such as find an exercise, was it clear on what to do?
	\item Considering the navigation of this program was it hard to locate various operations and actions from any given point?
\end{enumerate}

\noindent As for the non-user tests, user tests were preformed in order to measure metrics based on the application performance.
Ratings given were on a 5 point scale.

\noindent The summary of the test cases are displayed below.\\
Each test includes the outcome of our user testing and a unique identifier connecting these results to the tests outlined in the VnV Plan.\\

\subsection{Look and Feel Testing}
\begin{enumerate}
	\item{\textbf{test-LF-1}}: Style.\\
	\textbf{Test result}: 
	
	\begin{center}
		\begin{tabular}{ | m{3cm} | m{3cm}| m{6cm} | } 
		  \hline
		  User & Rating & Notes \\ 
		  \hline
		  User 1 & 3 & Color scheme is reminiscent of other popular apps. \\ 
		  \hline
		  User 2 & 5 & None. \\ 
		  \hline
		  User 3 & 5 & None. \\ 
		  \hline
		\end{tabular}
	\end{center}

\subsubsection{Usability and Humanity Tests}
	\item{\textbf{test-UH-1}}: Text Sizing and Visibility.\\
	\textbf{Test result}:
	\begin{center}
		\begin{tabular}{ | m{3cm} | m{3cm}| m{6cm} | } 
		  \hline
		  User & Rating & Notes \\ 
		  \hline
		  User 1 & 5 & None.\\ 
		  \hline
		  User 2 & 5 & None.\\ 
		  \hline
		  User 3 & 5 & None. \\ 
		  \hline
		\end{tabular}
	\end{center}
	
	\item{\textbf{test-UH-2}}: Text Language.\\
	\textbf{Test result}:
	N/A
	
	\item{\textbf{test-UH-3}}: Learnability.\\
	\textbf{Test result}:
	The task given to the users was to create and edit a workout program.
	\begin{center}
		\begin{tabular}{ | m{3cm} | m{3cm}| m{6cm} | } 
		  \hline
		  User & Rating & Notes \\ 
		  \hline
		  User 1 & 5 & \\ 
		  \hline
		  User 2 & 4 & I would appreciate it if I could extend my workout off of some default options rather than making it from scratch. \\ 
		  \hline
		  User 3 & 5 & The process was really easy and I liked seeing it in my program once I was done. \\ 
		  \hline
		\end{tabular}
	\end{center}
	
	\item{\textbf{test-UH-4}}: Understandability.\\
	\textbf{Test result}:
	Users were shown the user profile, program, and exercise pages and asked questions about what each page displays.
	\begin{center}
		\begin{tabular}{ | m{3cm} | m{3cm}| m{6cm} | } 
		  \hline
		  User & Rating & Notes \\ 
		  \hline
		  User 1 & 2 & It is hard to understand how programs are connected to exercises. \\ 
		  \hline
		  User 2 & 5 & None. \\ 
		  \hline
		  User 3 & 5 & None. \\ 
		  \hline
		\end{tabular}
	\end{center}
	
	\item{\textbf{test-UH-5}}: Hearing and Audio considerations.\\
	\textbf{Test result}:
	N/A
	
	\item{\textbf{test-UH-6}}: Use of Colour and Contrast.\\
	\textbf{Test result}:
	\begin{center}
		\begin{tabular}{ | m{3cm} | m{3cm}| m{6cm} | } 
		  \hline
		  User & Rating & Notes \\ 
		  \hline
		  User 1 & 2 & Color scheme is boring. \\ 
		  \hline
		  User 2 & 5 & Very clean design and informative use of color. \\ 
		  \hline
		  User 3 & 4 & Would appreciate a dark mode.\\ 
		  \hline
		\end{tabular}
	\end{center}
	
\subsubsection{Performance Tests}
	\item{\textbf{test-PF-1}}: Speed and Latency.\\
	\textbf{Test result}: Did not utilize a testing framework.
	
	\item{\textbf{test-PF-2}}: Accuracy and Precision of Quantifiers.\\
	\textbf{Test result}: [Passed]
	
	\item{\textbf{test-PF-3}}: Availability and Uptime.\\
	\textbf{Test result}: [Passed]
	
	\item{\textbf{test-PF-4}}: User Capacity.\\
	\textbf{Test result}: [Passed]
	
	\item{\textbf{test-PF-5}}: Scalability of User Capacity.\\
	\textbf{Test result}: [Passed]
	
\subsubsection{Operational and Environment Tests}
	\item{\textbf{test-OE-1}}: Supported Systems.\\
	\textbf{Test result}: N/A
	
\subsubsection{Maintainability and Support Tests}
	\item{\textbf{test-MS-1}}: Maintenance.\\
	\textbf{Test result}: [Passed]
	
\subsubsection{Security Tests}
	\item{\textbf{test-SEC-1}}: Private and Public Details.\\
	\textbf{Test result}: [Passed]
	
	\item{\textbf{test-SEC-2}}: Passwords.\\
	\textbf{Test result}: [Passed]
	
	\item{\textbf{test-SEC-3}}: Client Server Privacy.\\
	\textbf{Test result}: [Passed]
	
	\item{\textbf{test-SEC-4}}: Data storage and logging.\\
	\textbf{Test result}: [Passed]
	
	\item{\textbf{test-SEC-5}}: Data Backups.\\
	\textbf{Test result}: [Passed]
	
\subsubsection{Cultural Requirements Tests}
	\item{\textbf{test-CR-1}}: Profanity and Inappropriate Language.\\
	\textbf{Test result}: [Passed]
	
	\item{\textbf{test-CR-2}}: Reporting Offensive Language.\\
	\textbf{Test result}: [Passed]
	
\subsubsection{Legal Requirements Tests}
	\item{\textbf{test-LR-1}}: Age and Gender Use.\\
	\textbf{Test result}: [Passed]
	
	\item{\textbf{test-LR-2}}: Data Protection.\\
	\textbf{Test result}: [Passed]
	
\end{enumerate}

\section{Unit Testing}
\begin{enumerate}
	\item Test: UT-1 \\
	Purpose: Verify that the passwords are hashed \\
	Input: "hello" (String representing the password)\\
	Expected Output: A random hash\\
	Test Result:  Output was a hash and not the inputted password [Passed]\\

  \item Test: UT-2 \\
	Purpose: To check whether a user sign up accurately adds a user to the database\\
	Input: User sign up fields (name, password, email, username)\\
	Expected Output: signUpUser function returns true indicating user is signed up\\
	Test Result:  Output was true indicating user was signed up [Passed]\\

  \item Test: UT-3 \\
	Purpose: To verify that an incorrect username input won't login a user \\
	Input: Incorrect username into login field \\
	Expected Output: Null user token\\
	Test Result: Null user token [Passed]\\

  \item Test: UT-4 \\
	Purpose: To verify that a correct username input will login a user \\
	Input: Correct username into login field \\
	Expected Output: User token\\
	Test Result: User token [Passed]\\
\end{enumerate}

\section{Changes Due to Testing}
Listed below are some of the changes that are planned (some implemented) as a result of the testing and user testing.

\subsection{Plan for failed tests}
For any failed tests, we will review the specific "pain points" that the user experienced while trying to accomplish their task. We will focus our development work to target these pain points and fix any issues that may be present, whether they be functional or nonfunctional in nature. \\ \\
The objective is that the user would not encounter those issues they described if they were to be re-tested with the final implementation of the application.

\subsection{Timer}
During user testing one of the most desired additions to this application was a built in timer that would help for rest period and counting during workout sets. This addition makes sense as it helps users keep track of their timings without having the close the application to open up a timer. 
\subsection{Auto Login}
Another user suggestion was that when logging into the application from a point where the user has already logged in (e.g. logged in then closed the application then open again) then it is 'tedious' to log in again. There were many reasons for this such as users closing the application to change their music or check a text etc. Overall this change will help with the overall efficiency and user experience by bringing a more user friendly environment with less user actions in order to login and continue working out.
\subsection{Styling}
There were many positive reviews with the looks and styling of the application, However one of the minor changes that was mentioned was that some of the designed were 'too close together'. Taking in this feedback, the visual components may need to be spaced out more or even re-formatted in such a way that everything is less squashed together.
\subsection{Tracking user Fitness Goals}
Tracking Fitness goals  has always been a stretch goal of this application. Some users asked to see a visual representation such as a graph over time or a numerical increment to help them see progress. This change will help users visually see and track their fitness goals and progress easier.

\section{Trace to Requirements}
Below is the link from requirements to test cases.
\begin{table}[H]
	\centering
	\begin{tabular}{|c|c|c|c|c|c|c|c|c|c|c|c|}
		\hline
		& R1 & R2 & R3 & R4 & R5 & R6 & R7 & R8 & R9 & R10 & R11 \\ \hline
		WR1 &X & & & & & & & & & & \\ \hline
		WR2 &X & & & & & & & & & & \\ \hline
		EX1 & &X & & & & & & & & & \\ \hline 
		EX2 & &X & & & & & & & & & \\ \hline
		EX3 & &X & & & & & & & & &\\ \hline 
		QT1 & & &X & & & & & & & & \\ \hline 
		QT2 & & &X & & & & & & & & \\ \hline 
		QT3 & & &X & & & & & & & & \\ \hline 
		PB1 & & & &X & & & & & & & \\ \hline 
		PB2 & & & &X & & & & & & & \\ \hline 
		WS1 & & & & &X & & & & & & \\ \hline 
		BS1 & & & & & &X & & & & & \\ \hline 
		BS2 & & & & & &X & & & & & \\ \hline 
		UP1 & & & & & & &X & & & & \\ \hline 
		UP2 & & & & & & & &X & & & \\ \hline 
		FG1 & & & & & & & & &X & & \\ \hline 
		FG2 & & & & & & & & & &X &X \\ \hline 	
	\end{tabular}
	\caption{Functional System Tests to Functional Requirement Matrix}
	\label{Table:R_trace}
\end{table}

\begin{table}[H]
	\begin{adjustwidth}{-3cm}{-5cm}
		\begin{tabular}{|c|c|c|c|c|c|c|c|c|c|c|c|}
			\hline
			& NFR1 & NFR2 & NFR3 & NFR4 & NFR5 & NFR6 & NFR7 & NFR8 & NFR9 & NFR10 & NFR11 \\ \hline
			LF1 &X & & & & & & & & & & \\ \hline
			UH1 & &X & & & & & & & & & \\ \hline
			UH2 & & &X & & & & & & & & \\ \hline
			UH3 & & & &X & & & & & & & \\ \hline 
			UH4 & & & & &X & & & & & & \\ \hline 
			UH5 & & & & & &X & & & & & \\ \hline 
			UH6 & & & & & & &X & & & & \\ \hline 
			PF1 & & & & & & & &X & & & \\ \hline 
			PF2 & & & & & & & & &X & & \\ \hline 
			PF3 & & & & & & & & & &X & \\ \hline
			PF4 & & & & & & & & & & &X \\ \hline  
		\end{tabular}
		\caption{Non-Functional System Tests to Non-Functional Requirement Matrix}
		\label{Table:R_trace}
		
		\begin{tabular}{|c|c|c|c|c|c|c|c|c|c|c|c|}
			\hline
			& NFR12 & NFR13 & NFR14 & NFR15 & NFR16 & NFR17 & NFR18 & NFR19 & NFR20 & NFR21 & NFR22 \\ \hline
			PF5 &X & & & & & & & & & & \\ \hline
			OE1 & &X & & & & & & & & & \\ \hline
			MS1 & & &X & & & & & & & & \\ \hline
			SEC1 & & & &X & & & & & & & \\ \hline 
			SEC2 & & & & &X & & & & & & \\ \hline 
			SEC3 & & & & & &X & & & & & \\ \hline 
			SEC4 & & & & & &X & & & & & \\ \hline 
			SEC5 & & & & & & &X & & & & \\ \hline 
			CR1 & & & & & & & &X & & & \\ \hline 
			CR2 & & & & & & & & &X & & \\ \hline 
			LR1 & & & & & & & & & &X & \\ \hline 
			LR2 & & & & & & & & & & &X \\ \hline 
		\end{tabular}
		\caption{Non-Functional System Tests to Non-Functional Requirement Matrix}
		\label{Table:R_trace}
	\end{adjustwidth}
\end{table}
\section{Trace to Modules}		
\begin{table}[H]
	\begin{tabular}{|c|c|c|}
		\hline
		\textbf{Module Name} & \textbf{Associated Test Ids}\\ \hline
		Exercise Module & EX-1, QT-1, QT-2, QT-3  \\\hline
		Workout Routine Module & WR-1, EX-1, EX-2, EX-3, PB-1, PB-2, WS-1\\\hline
		User Login Module & SEC-2, UT-1, UT-3, UT-4 \\\hline
		User Registration Module & SEC-2, UT-1, UT-2\\\hline
		User Profile Module & WS-1, UP-1, UP-2, FG-1, FG-2\\\hline
		User Fitness Goal Module & FG-1, FG-2\\\hline
		Workout Browser Module & PB-1, PB-2, WS-1, BS-1, BS-2\\\hline
		Creation Module & WR-1, EX-1 \\\hline
		Workout Routine Creation Module & WR-1,WR-2\\\hline
		Exercise Creation Module & EX-1, QT-1\\\hline
		Database Communicator Module & SEC-4, SEC-5, LR-2\\\hline
		
	\end{tabular}
	\caption{Test to Module Traceability Matrix}
	\label{Table:R_trace}
\end{table}
\section{Code Coverage Metrics}

\bibliographystyle{plainnat}

\newpage{}
\section*{Appendix --- Reflection}

The information in this section will be used to evaluate the team members on the
graduate attribute of Lifelong Learning.  Please answer the following questions:

\begin{enumerate}
	\item \textbf{Question} 
	
	In what ways was the Verification and Validation (VnV) Plan different
	from the activities that were actually conducted for VnV?  If there were
	differences, what changes required the modification in the plan?  Why did
	these changes occur?  Would you be able to anticipate these changes in future
	projects?  If there weren't any differences, how was your team able to clearly
	predict a feasible amount of effort and the right tasks needed to build the
	evidence that demonstrates the required quality?  (It is expected that most
	teams will have had to deviate from their original VnV Plan.)
	
	\item \textbf{Answer}
	
	The VnV plan differed from the actual activities that were performed in many ways, most notably the user testing. We underestimated the amount of knowledge that our application actually conveyed to the user. This oversight was likely due to the common pitfall of failing to explain details that the developers were very familiar with. As a result, testers often had to direct users and answer questions regarding the performance of specific tasks, that were otherwise meant to be unguided. Knowing the application lacked intuitive flow was valuable information, sparked further revisions to the UX.
	 
	In the VnV plan we stated that we would conduct surveys on some of the 'Aesthetics' of the design, however this was very tedious to do as a tester so we ended up asking the individuals we tested on about the look and feel of the application. Another way we deviated from the VnV plan was that large data-management testing is very difficult and takes some time. We were not able to complete certain tests such as data backups due to the complexity that this requires.
	
	To improve future verification and validation, more user testing can be done. Opinions of people are subjective, and often vary wildly. In order to be convinced that given functionality meets criteria on effectiveness, ease of use etc, a significant number of data points (in the form of user tests) are required.
	
	 One of the major lessons learned was that verification and validation takes time. It is easy to fall into the trap of thinking something you wrote should work and function as intended, but this is hardly ever the case over the scope of an entire project. Deviations from the VnV Plan in the form of shortcuts could be avoided with better planning and time allocation.
	
\end{enumerate}

\newpage{}
\section{Appendix: User testing Results}
\subsection{User 1}
Actions and results
\begin{enumerate}
\item Action: Creating an account

Result: [Passed] User successfully found path to creating an account without asking any questions.

-	User liked that it was obvious what to do at each point
-	User liked that they could undo actions and easily fix any errors with their password / email.

\item Action: Logging into an account.

Result: [Passed] User logged in without and issues

-	User suggested that they should be automatically logged in once creating an account
-	User liked the quickness of logging in

\item Action: Creating a Program.

Result: [Passed] 

-	Issues: no ‘create program button’ but there was a get started one which worked
-	Liked the Tags
-	Liked the simplicity of creating a program

\item Action: Creating a Workout.

Result: [Passed]

-	Liked the muscle categories with the visual

-	Disliked repetition of adding exercises

\item Action: Creating an Exercise.

Result: [Passed]

-	Liked the muscle visual

\item Action: Searching for an Exercise.

Result: [Passed]

-	Liked the muscle visuals (again)

-	Liked the auto addition to workout

\item Action: Using the Discovery page to browse workouts and programs.

Result: [Passed]

-	Liked the browse page categories

-	Liked the simplicity in scrolling and searching


\item Action: Starting a workout (as if the user were to use the application during their workout).

Result: [Passed]

-	User liked the workout page's "play button" as an intuitive button to start the workout
\end{enumerate}

\subsection{User 2}
Actions and results
\begin{enumerate}
	\item Action: Creating an account
	
	Result: [Passed] User was able to successfully create an account by following the instructions provided at each stage
	
	-	User liked that they were able to submit their textbox by either hitting the "submit" button on their keyboard, or by tapping outside of the keyboard.
	-	User liked that the account creation process was minimal and did not ask for excessive information.
	
	\item Action: Logging into an account.
	
	Result: [Passed] User logged in without and issues
	
	-	User suggested a "remember me" option for keeping the username field populated with their username on their device.
	-	User was able to quickly log in.
	
	\item Action: Creating a Program.
	
	Result: [Passed] 
	
	-	Issues: Did not like being limited by just 4 tags
	-	Liked the feature of controlling the publicity of their program
	-	Liked the smoothness of the program creation process.
	
	\item Action: Creating a Workout.
	
	Result: [Passed]
	
	-	Was able to start the workout without any assistance
	
	\item Action: Creating an Exercise.
	
	Result: [Passed]
	
	-	Liked the muscle visualization.
	
	- Suggested that a similar visualization could be done for cardio exercises, providing imagery for the user when selecting different types of cardio exercises.
	
	\item Action: Searching for an Exercise.
	
	Result: [Passed]
	
	- Liked the straight-forward search process.
	
	\item Action: Using the Discovery page to browse workouts and programs.
	
	Result: [Passed]
	
	-	Liked the preset categories
	
	-	Liked the trending programs display category
	
	-	Said it reminded them of the "Explore" page from Instagram
	
	
	\item Action: Starting a workout (as if the user were to use the application during their workout).
	
	Result: [Passed]
	
	-	User had no trouble starting their workout and following the instructions
	
\item Other: 

- User Suggested a more wow-ing colour scheme
\end{enumerate}

\subsection{User 3}
Actions and results
\begin{enumerate}
\item Action: Creating an account

Result:  [Passed]

-	User stated that this was a very standard login (not that anything is wrong with that) and that they liked that we followed similar conventions

\item Action: Logging into an account.

Result: [Passed]

-	User suggested that the application logged in automatically after they created an account because it feels like it takes a while to even ‘enter’ the application


\item Action: Creating a Program.

Result: [Passed]

-	User Liked the simplicity and customization of the programs.

-	User emphasised that they like the personalization and wishes to see even more such as suggestions for workouts based on their fitness goals.


\item Action: Creating a Workout.

Result: [Passed]

-	User said that this was exactly what they expected 

\item Action: Creating an Exercise.

Result: [Passed]

-	User liked the muscle visualizations and the little details into the primary and secondary muscles used in an exercise.

-	User suggested that we make it more clear on certain muscle targets for different angles of a workout that could ‘hit’ different muscles


\item Action: Searching for an Exercise.

Result: [Passed]

-	User made a similar note on the muscle visualizations

\item Action: Using the Discovery page to browse workouts and programs.

Result: [Passed]

-	User found it fairly simple

-	User liked the different workout tailored tasks such as ‘yoga’ or 


\item Action: Starting a workout (as if the user were to use the application during their workout)
Result:  [Passed]

-	User was wondering if there was a way to time their workouts for stuff like ’30 seconds of crunches’

-	User liked the easy ‘flow’ of the workout

\item Other: 

- User requested a dark mode

\end{enumerate}

\end{document}