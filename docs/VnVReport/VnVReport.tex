\documentclass[12pt, titlepage]{article}

\usepackage{booktabs}
\usepackage{tabularx}
\usepackage{hyperref}
\usepackage{changepage}
\usepackage{float}
\hypersetup{
    colorlinks,
    citecolor=black,
    filecolor=black,
    linkcolor=red,
    urlcolor=blue
}
\usepackage[round]{natbib}

\input{../Comments}
\input{../Common}

\begin{document}

\title{Verification and Validation Report: \progname} 
\author{\authname}
\date{\today}
	
\maketitle

\pagenumbering{roman}

\section{Revision History}

\begin{tabularx}{\textwidth}{p{3cm}p{2cm}X}
\toprule {\bf Date} & {\bf Version} & {\bf Notes}\\
\midrule
Date 1 & 1.0 & Notes\\
Date 2 & 1.1 & Notes\\
\bottomrule
\end{tabularx}

~\newpage

\section{Symbols, Abbreviations and Acronyms}

\renewcommand{\arraystretch}{1.2}
\begin{tabular}{l l} 
  \toprule		
  \textbf{symbol} & \textbf{description}\\
  \midrule 
  T & Test\\
  \bottomrule
\end{tabular}\\

\wss{symbols, abbreviations or acronyms -- you can reference the SRS tables if needed}

\newpage

\tableofcontents

\listoftables %if appropriate

\listoffigures %if appropriate

\newpage

\pagenumbering{arabic}

This document ...

\section{User Testing Summary}
Unless Specified these tests were performed with a semi-structured interview where users were brought scenarios then asked questions relevant to that scenario. Most Scenarios followed the key use cases of the application.\\
Examples include:
\begin{enumerate}
	\item Creating an Account.
	\item Logging into an account.
	\item Creating a Program.
	\item Creating a Workout.
	\item Creating an Exercise.
	\item Searching for an Exercise.
	\item Using the Discovery page to browse workouts and programs.
	\item Finding new workouts from the Discovery page.
	\item Starting a workout (as if the user were to use the application during their workout).
\end{enumerate}


\section{Functional Requirements Evaluation}

\noindent During the user tests, users were asked to preform extra actions based on the FR test cases displayed in the VnV Plan.
Some example actions included: \\
\begin{enumerate}
	\item During workout creation, could you make your workout private?
	\item Could you try failing your password?
	\item Asking the user to attempt all actions when starting a workout.
\end{enumerate}

\noindent As for the non-user tests, unit tests were preformed in order to measure metrics.

\noindent The summary of the test cases are displayed below.

\begin{enumerate}
	
\subsubsection{Workout Routine Tests}
	\item{\textbf{test-WR-1}}:A Workout Routine can be created.\\
	\textbf{Test result}:
	
	\item{\textbf{test-WR-2}}: Editing a Workout Routine.\\
	\textbf{Test result}:
	
\subsubsection{Exercise Tests}
	\item{\textbf{test-EX-1}}:Adding an Exercise to a Workout Routine.\\
	\textbf{Test result}:
	
	\item{\textbf{test-EX-2}}:Removing an Exercise from a Workout Routine.\\
	\textbf{Test result}:
	
	\item{\textbf{test-EX-3}}:Limiting Exercises to a Workout Routine.\\
	\textbf{Test result}:
	
\subsubsection{Quantifier Tests}
	\item{\textbf{test-QT-1}}:Adding Quantifiers to an Exercise.\\
	\textbf{Test result}:
	
	\item{\textbf{test-QT-2}}:Removing Quantifiers from an Exercise.\\
	\textbf{Test result}:
	
	\item{\textbf{test-QT-3}}:Editing Quantifiers of an Exercise.\\
	\textbf{Test result}:
	
\subsubsection{Publicity Tests}
	\item{\textbf{test-PB-1}}:Publicizing a Workout Routine.\\
	\textbf{Test result}:
	
	\item{\textbf{test-PB-2}}:Privatizing a Workout Routine.\\
	\textbf{Test result}:
	
\subsubsection{Workout Routine Saving Tests}
	\item{\textbf{test-WS-1}}:Saving a Public Workout Routine.\\
	\textbf{Test result}:
	
\subsubsection{Browsing Workout Routine Tests}
	\item{\textbf{test-BS-1}}:Browsing Workout Routines.\\
	\textbf{Test result}:
	
	\item{\textbf{test-BS-2}}:Search Workout Routines.\\
	\textbf{Test result}:
	
\subsubsection{User Profile Tests}
	\item{\textbf{test-UP-1}}:Creating a User Profile.\\
	\textbf{Test result}:
	
	\item{\textbf{test-UP-2}}:Viewing Other User Profile.\\
	\textbf{Test result}:
	
\subsubsection{Fitness Goal Tests}
	\item{\textbf{test-FG-1}}:Creating a Fitness Goal.\\
	\textbf{Test result}:
	
	\item{\textbf{test-FG-2}}:Progressing a Fitness Goal.\\
	\textbf{Test result}:
	
	
\end{enumerate}
\section{Nonfunctional Requirements Evaluation}

\noindent After users were brought through the test case scenarios they were asked questions based on the NFR test cases displayed in the VnV Plan.
Some example questions included: \\
\begin{enumerate}
	\item How did you find the speed of the application? Did you notice any latency or lagging?
	\item Any time the application prompted you to do an action such as find an exercise, was it clear on what to do?
	\item Considering the navigation of this program was it hard to locate various operations and actions from any given point?
\end{enumerate}

\noindent As for the non-user tests, unit tests were preformed in order to measure metrics based on the application performance.

\noindent The summary of the test cases are displayed below.

\subsection{Look and Feel Testing}
\begin{enumerate}
	\item{\textbf{test-LF-1}}: Style.\\
	\textbf{Test result}:
	
\subsubsection{Usability and Humanity Tests}
	\item{\textbf{test-UH-1}}: Text Sizing and Visibility.\\
	\textbf{Test result}:
	
	\item{\textbf{test-UH-2}}: Text Language.\\
	\textbf{Test result}:
	
	\item{\textbf{test-UH-3}}: Learnability.\\
	\textbf{Test result}:
	
	\item{\textbf{test-UH-4}}: Understandability.\\
	\textbf{Test result}:
	
	\item{\textbf{test-UH-5}}: Hearing and Audio considerations.\\
	\textbf{Test result}:
	
	\item{\textbf{test-UH-6}}: Use of Colour and Contrast.\\
	\textbf{Test result}:
	
\subsubsection{Performance Tests}
	\item{\textbf{test-PF-1}}: Speed and Latency.\\
	\textbf{Test result}:
	
	\item{\textbf{test-PF-2}}: Accuracy and Precision of Quantifiers.\\
	\textbf{Test result}:
	
	\item{\textbf{test-PF-3}}: Availability and Uptime.\\
	\textbf{Test result}:
	
	\item{\textbf{test-PF-4}}: User Capacity.\\
	\textbf{Test result}:
	
	\item{\textbf{test-PF-5}}: Scalability of User Capacity.\\
	\textbf{Test result}:
	
\subsubsection{Operational and Environment Tests}
	\item{\textbf{test-OE-1}}: Supported Systems.\\
	\textbf{Test result}:
	
\subsubsection{Maintainability and Support Tests}
	\item{\textbf{test-MS-1}}: Maintenance.\\
	\textbf{Test result}:
	
\subsubsection{Security Tests}
	\item{\textbf{test-SEC-1}}: Private and Public Details.\\
	\textbf{Test result}:
	
	\item{\textbf{test-SEC-2}}: Passwords.\\
	\textbf{Test result}:
	
	\item{\textbf{test-SEC-3}}: Client Server Privacy.\\
	\textbf{Test result}:
	
	\item{\textbf{test-SEC-4}}: Data storage and logging.\\
	\textbf{Test result}:
	
	\item{\textbf{test-SEC-5}}: Data Backups.\\
	\textbf{Test result}:
	
\subsubsection{Cultural Requirements Tests}
	\item{\textbf{test-CR-1}}: Profanity and Inappropriate Language.\\
	\textbf{Test result}:
	
	\item{\textbf{test-CR-2}}: Reporting Offensive Language.\\
	\textbf{Test result}:
	
\subsubsection{Legal Requirements Tests}
	\item{\textbf{test-LR-1}}: Age and Gender Use.\\
	\textbf{Test result}:
	
	\item{\textbf{test-LR-2}}: Data Protection.\\
	\textbf{Test result}:
	
\end{enumerate}
	
\section{Comparison to Existing Implementation}	

This section will not be appropriate for every project.

\section{Unit Testing}

\section{Changes Due to Testing}
Listed below are some of the changes that are planned (some implemented) as a result of the testing and user testing.
\subsection{Timer}
During user testing one of the most desired additions to this application was a built in timer that would help for rest period and counting during workout sets. This addition makes sense as it helps users keep track of their timings without having the close the application to open up a timer. 
\subsection{Auto Login}
Another user suggestion was that when logging into the application from a point where the user has already logged in (e.g. logged in then closed the application then open again) then it is 'tedious' to log in again. There were many reasons for this such as users closing the application to change their music or check a text etc. Overall this change will help with the overall efficiency and user experience by bringing a more user friendly environment with less user actions in order to login and continue working out.
\subsection{Styling}
There were many positive reviews with the looks and styling of the application, However one of the minor changes that was mentioned was that some of the designed were 'too close together'. Taking in this feedback, the visual components may need to be spaced out more or even re-formatted in such a way that everything is less squashed together.
\subsection{Tracking user Fitness Goals}
Tracking Fitness goals has always been a stretch goal of this application. Some users asked to see a visual representation such as a graph over time or a numerical increment to help them see progress. This change will help users visually see and track their fitness goals and progress easier.

\section{Automated Testing}
		
\section{Trace to Requirements}
Below is the link from requirements to test cases.
\begin{table}[H]
	\centering
	\begin{tabular}{|c|c|c|c|c|c|c|c|c|c|c|c|}
		\hline
		& R1 & R2 & R3 & R4 & R5 & R6 & R7 & R8 & R9 & R10 & R11 \\ \hline
		WR1 &X & & & & & & & & & & \\ \hline
		WR2 &X & & & & & & & & & & \\ \hline
		EX1 & &X & & & & & & & & & \\ \hline 
		EX2 & &X & & & & & & & & & \\ \hline
		EX3 & &X & & & & & & & & &\\ \hline 
		QT1 & & &X & & & & & & & & \\ \hline 
		QT2 & & &X & & & & & & & & \\ \hline 
		QT3 & & &X & & & & & & & & \\ \hline 
		PB1 & & & &X & & & & & & & \\ \hline 
		PB2 & & & &X & & & & & & & \\ \hline 
		WS1 & & & & &X & & & & & & \\ \hline 
		BS1 & & & & & &X & & & & & \\ \hline 
		BS2 & & & & & &X & & & & & \\ \hline 
		UP1 & & & & & & &X & & & & \\ \hline 
		UP2 & & & & & & & &X & & & \\ \hline 
		FG1 & & & & & & & & &X & & \\ \hline 
		FG2 & & & & & & & & & &X &X \\ \hline 	
	\end{tabular}
	\caption{Functional System Tests to Functional Requirement Matrix}
	\label{Table:R_trace}
\end{table}

\begin{table}[H]
	\begin{adjustwidth}{-3cm}{-5cm}
		\begin{tabular}{|c|c|c|c|c|c|c|c|c|c|c|c|}
			\hline
			& NFR1 & NFR2 & NFR3 & NFR4 & NFR5 & NFR6 & NFR7 & NFR8 & NFR9 & NFR10 & NFR11 \\ \hline
			LF1 &X & & & & & & & & & & \\ \hline
			UH1 & &X & & & & & & & & & \\ \hline
			UH2 & & &X & & & & & & & & \\ \hline
			UH3 & & & &X & & & & & & & \\ \hline 
			UH4 & & & & &X & & & & & & \\ \hline 
			UH5 & & & & & &X & & & & & \\ \hline 
			UH6 & & & & & & &X & & & & \\ \hline 
			PF1 & & & & & & & &X & & & \\ \hline 
			PF2 & & & & & & & & &X & & \\ \hline 
			PF3 & & & & & & & & & &X & \\ \hline
			PF4 & & & & & & & & & & &X \\ \hline  
		\end{tabular}
		\caption{Non-Functional System Tests to Non-Functional Requirement Matrix}
		\label{Table:R_trace}
		
		\begin{tabular}{|c|c|c|c|c|c|c|c|c|c|c|c|}
			\hline
			& NFR12 & NFR13 & NFR14 & NFR15 & NFR16 & NFR17 & NFR18 & NFR19 & NFR20 & NFR21 & NFR22 \\ \hline
			PF5 &X & & & & & & & & & & \\ \hline
			OE1 & &X & & & & & & & & & \\ \hline
			MS1 & & &X & & & & & & & & \\ \hline
			SEC1 & & & &X & & & & & & & \\ \hline 
			SEC2 & & & & &X & & & & & & \\ \hline 
			SEC3 & & & & & &X & & & & & \\ \hline 
			SEC4 & & & & & &X & & & & & \\ \hline 
			SEC5 & & & & & & &X & & & & \\ \hline 
			CR1 & & & & & & & &X & & & \\ \hline 
			CR2 & & & & & & & & &X & & \\ \hline 
			LR1 & & & & & & & & & &X & \\ \hline 
			LR2 & & & & & & & & & & &X \\ \hline 
		\end{tabular}
		\caption{Non-Functional System Tests to Non-Functional Requirement Matrix}
		\label{Table:R_trace}
	\end{adjustwidth}
\end{table}
\section{Trace to Modules}		

\section{Code Coverage Metrics}

\bibliographystyle{plainnat}
\bibliography{../../refs/References}

\newpage{}
\section*{Appendix --- Reflection}

The information in this section will be used to evaluate the team members on the
graduate attribute of Lifelong Learning.  Please answer the following questions:

\begin{enumerate}
  \item 
  \item 
\end{enumerate}

\end{document}